% Options for packages loaded elsewhere
\PassOptionsToPackage{unicode}{hyperref}
\PassOptionsToPackage{hyphens}{url}
%
\documentclass[
]{article}
\usepackage{lmodern}
\usepackage{amsmath}
\usepackage{ifxetex,ifluatex}
\ifnum 0\ifxetex 1\fi\ifluatex 1\fi=0 % if pdftex
  \usepackage[T1]{fontenc}
  \usepackage[utf8]{inputenc}
  \usepackage{textcomp} % provide euro and other symbols
  \usepackage{amssymb}
\else % if luatex or xetex
  \usepackage{unicode-math}
  \defaultfontfeatures{Scale=MatchLowercase}
  \defaultfontfeatures[\rmfamily]{Ligatures=TeX,Scale=1}
\fi
% Use upquote if available, for straight quotes in verbatim environments
\IfFileExists{upquote.sty}{\usepackage{upquote}}{}
\IfFileExists{microtype.sty}{% use microtype if available
  \usepackage[]{microtype}
  \UseMicrotypeSet[protrusion]{basicmath} % disable protrusion for tt fonts
}{}
\makeatletter
\@ifundefined{KOMAClassName}{% if non-KOMA class
  \IfFileExists{parskip.sty}{%
    \usepackage{parskip}
  }{% else
    \setlength{\parindent}{0pt}
    \setlength{\parskip}{6pt plus 2pt minus 1pt}}
}{% if KOMA class
  \KOMAoptions{parskip=half}}
\makeatother
\usepackage{xcolor}
\IfFileExists{xurl.sty}{\usepackage{xurl}}{} % add URL line breaks if available
\IfFileExists{bookmark.sty}{\usepackage{bookmark}}{\usepackage{hyperref}}
\hypersetup{
  pdftitle={Probability Notes},
  pdfauthor={Phillip Escandon - escandon@bu.edu},
  hidelinks,
  pdfcreator={LaTeX via pandoc}}
\urlstyle{same} % disable monospaced font for URLs
\usepackage[margin=1in]{geometry}
\usepackage{color}
\usepackage{fancyvrb}
\newcommand{\VerbBar}{|}
\newcommand{\VERB}{\Verb[commandchars=\\\{\}]}
\DefineVerbatimEnvironment{Highlighting}{Verbatim}{commandchars=\\\{\}}
% Add ',fontsize=\small' for more characters per line
\usepackage{framed}
\definecolor{shadecolor}{RGB}{248,248,248}
\newenvironment{Shaded}{\begin{snugshade}}{\end{snugshade}}
\newcommand{\AlertTok}[1]{\textcolor[rgb]{0.94,0.16,0.16}{#1}}
\newcommand{\AnnotationTok}[1]{\textcolor[rgb]{0.56,0.35,0.01}{\textbf{\textit{#1}}}}
\newcommand{\AttributeTok}[1]{\textcolor[rgb]{0.77,0.63,0.00}{#1}}
\newcommand{\BaseNTok}[1]{\textcolor[rgb]{0.00,0.00,0.81}{#1}}
\newcommand{\BuiltInTok}[1]{#1}
\newcommand{\CharTok}[1]{\textcolor[rgb]{0.31,0.60,0.02}{#1}}
\newcommand{\CommentTok}[1]{\textcolor[rgb]{0.56,0.35,0.01}{\textit{#1}}}
\newcommand{\CommentVarTok}[1]{\textcolor[rgb]{0.56,0.35,0.01}{\textbf{\textit{#1}}}}
\newcommand{\ConstantTok}[1]{\textcolor[rgb]{0.00,0.00,0.00}{#1}}
\newcommand{\ControlFlowTok}[1]{\textcolor[rgb]{0.13,0.29,0.53}{\textbf{#1}}}
\newcommand{\DataTypeTok}[1]{\textcolor[rgb]{0.13,0.29,0.53}{#1}}
\newcommand{\DecValTok}[1]{\textcolor[rgb]{0.00,0.00,0.81}{#1}}
\newcommand{\DocumentationTok}[1]{\textcolor[rgb]{0.56,0.35,0.01}{\textbf{\textit{#1}}}}
\newcommand{\ErrorTok}[1]{\textcolor[rgb]{0.64,0.00,0.00}{\textbf{#1}}}
\newcommand{\ExtensionTok}[1]{#1}
\newcommand{\FloatTok}[1]{\textcolor[rgb]{0.00,0.00,0.81}{#1}}
\newcommand{\FunctionTok}[1]{\textcolor[rgb]{0.00,0.00,0.00}{#1}}
\newcommand{\ImportTok}[1]{#1}
\newcommand{\InformationTok}[1]{\textcolor[rgb]{0.56,0.35,0.01}{\textbf{\textit{#1}}}}
\newcommand{\KeywordTok}[1]{\textcolor[rgb]{0.13,0.29,0.53}{\textbf{#1}}}
\newcommand{\NormalTok}[1]{#1}
\newcommand{\OperatorTok}[1]{\textcolor[rgb]{0.81,0.36,0.00}{\textbf{#1}}}
\newcommand{\OtherTok}[1]{\textcolor[rgb]{0.56,0.35,0.01}{#1}}
\newcommand{\PreprocessorTok}[1]{\textcolor[rgb]{0.56,0.35,0.01}{\textit{#1}}}
\newcommand{\RegionMarkerTok}[1]{#1}
\newcommand{\SpecialCharTok}[1]{\textcolor[rgb]{0.00,0.00,0.00}{#1}}
\newcommand{\SpecialStringTok}[1]{\textcolor[rgb]{0.31,0.60,0.02}{#1}}
\newcommand{\StringTok}[1]{\textcolor[rgb]{0.31,0.60,0.02}{#1}}
\newcommand{\VariableTok}[1]{\textcolor[rgb]{0.00,0.00,0.00}{#1}}
\newcommand{\VerbatimStringTok}[1]{\textcolor[rgb]{0.31,0.60,0.02}{#1}}
\newcommand{\WarningTok}[1]{\textcolor[rgb]{0.56,0.35,0.01}{\textbf{\textit{#1}}}}
\usepackage{graphicx}
\makeatletter
\def\maxwidth{\ifdim\Gin@nat@width>\linewidth\linewidth\else\Gin@nat@width\fi}
\def\maxheight{\ifdim\Gin@nat@height>\textheight\textheight\else\Gin@nat@height\fi}
\makeatother
% Scale images if necessary, so that they will not overflow the page
% margins by default, and it is still possible to overwrite the defaults
% using explicit options in \includegraphics[width, height, ...]{}
\setkeys{Gin}{width=\maxwidth,height=\maxheight,keepaspectratio}
% Set default figure placement to htbp
\makeatletter
\def\fps@figure{htbp}
\makeatother
\setlength{\emergencystretch}{3em} % prevent overfull lines
\providecommand{\tightlist}{%
  \setlength{\itemsep}{0pt}\setlength{\parskip}{0pt}}
\setcounter{secnumdepth}{-\maxdimen} % remove section numbering
\ifluatex
  \usepackage{selnolig}  % disable illegal ligatures
\fi

\title{Probability Notes}
\author{Phillip Escandon -
\href{mailto:escandon@bu.edu}{\nolinkurl{escandon@bu.edu}}}
\date{01 February, 2021}

\begin{document}
\maketitle

{
\setcounter{tocdepth}{2}
\tableofcontents
}
\hypertarget{introduction}{%
\subsection{Introduction}\label{introduction}}

\begin{verbatim}
Using some of the common prob functions to solve the simple problems described in HW assignment 2.

1. Make a probabilty space
2. use PROB to define output.
3. What do intersect and union mean in this context
\end{verbatim}

\hypertarget{using-a-probability-space-and-describing-events}{%
\subsection{Using a probability space and describing
events}\label{using-a-probability-space-and-describing-events}}

\begin{Shaded}
\begin{Highlighting}[]
\CommentTok{\# Rolling  two dice }
\NormalTok{S}\OtherTok{\textless{}{-}} \FunctionTok{rolldie}\NormalTok{(}\DecValTok{2}\NormalTok{,}\AttributeTok{makespace =} \ConstantTok{TRUE}\NormalTok{)}
\FunctionTok{tibble}\NormalTok{(S)}
\end{Highlighting}
\end{Shaded}

\begin{verbatim}
## # A tibble: 36 x 3
##       X1    X2  probs
##    <int> <int>  <dbl>
##  1     1     1 0.0278
##  2     2     1 0.0278
##  3     3     1 0.0278
##  4     4     1 0.0278
##  5     5     1 0.0278
##  6     6     1 0.0278
##  7     1     2 0.0278
##  8     2     2 0.0278
##  9     3     2 0.0278
## 10     4     2 0.0278
## # ... with 26 more rows
\end{verbatim}

\begin{Shaded}
\begin{Highlighting}[]
\CommentTok{\# Now create a subset of what you\textquotesingle{}d like to check}
\NormalTok{A }\OtherTok{\textless{}{-}} \FunctionTok{subset}\NormalTok{(S,X1}\SpecialCharTok{==}\DecValTok{3}\NormalTok{)}
\FunctionTok{Prob}\NormalTok{(A)}
\end{Highlighting}
\end{Shaded}

\begin{verbatim}
## [1] 0.1666667
\end{verbatim}

\begin{Shaded}
\begin{Highlighting}[]
\CommentTok{\# or just use Prob directly:}

\FunctionTok{Prob}\NormalTok{(S, X1 }\SpecialCharTok{==} \DecValTok{3}\NormalTok{)}
\end{Highlighting}
\end{Shaded}

\begin{verbatim}
## [1] 0.1666667
\end{verbatim}

\begin{Shaded}
\begin{Highlighting}[]
\CommentTok{\# whats the prob of the first die = 4 or 5}
\FunctionTok{Prob}\NormalTok{(S, X1 }\SpecialCharTok{\%in\%} \FunctionTok{c}\NormalTok{(}\DecValTok{4}\NormalTok{,}\DecValTok{5}\NormalTok{))}
\end{Highlighting}
\end{Shaded}

\begin{verbatim}
## [1] 0.3333333
\end{verbatim}

\hypertarget{making-your-own-probabilty-space}{%
\subsection{Making your own probabilty
space}\label{making-your-own-probabilty-space}}

\begin{Shaded}
\begin{Highlighting}[]
\CommentTok{\# make my two columns 1. outcome, 2. probability and I set up myself}
\CommentTok{\# problem 2.8}
\NormalTok{outcomes}\OtherTok{\textless{}{-}} \FunctionTok{rolldie}\NormalTok{(}\DecValTok{1}\NormalTok{)}
\NormalTok{p}\OtherTok{\textless{}{-}} \FunctionTok{rep}\NormalTok{(}\DecValTok{1}\SpecialCharTok{/}\DecValTok{6}\NormalTok{, }\AttributeTok{times=}\DecValTok{6}\NormalTok{)}

\NormalTok{S}\OtherTok{\textless{}{-}}\FunctionTok{probspace}\NormalTok{(outcomes, }\AttributeTok{probs =}\NormalTok{ p)}
\FunctionTok{sum}\NormalTok{(S}\SpecialCharTok{$}\NormalTok{probs)}
\end{Highlighting}
\end{Shaded}

\begin{verbatim}
## [1] 1
\end{verbatim}

I can set up the probabilities at my own discretion, as long as they add
up to one. The sum of the above probabilities is 1

Now use it

\begin{Shaded}
\begin{Highlighting}[]
\CommentTok{\# to remind oneself of what dataframe we are dealing with}
\CommentTok{\# lets take a look}
\FunctionTok{head}\NormalTok{(S,}\DecValTok{2}\NormalTok{)}
\end{Highlighting}
\end{Shaded}

\begin{verbatim}
##   X1     probs
## 1  1 0.1666667
## 2  2 0.1666667
\end{verbatim}

\begin{Shaded}
\begin{Highlighting}[]
\CommentTok{\# We are looking at variables X1 and probs}
\FunctionTok{Prob}\NormalTok{(S,X1 }\SpecialCharTok{==} \DecValTok{4}\NormalTok{)}
\end{Highlighting}
\end{Shaded}

\begin{verbatim}
## [1] 0.1666667
\end{verbatim}

\begin{Shaded}
\begin{Highlighting}[]
\FunctionTok{Prob}\NormalTok{(S,X1 }\SpecialCharTok{\%in\%} \FunctionTok{c}\NormalTok{(}\DecValTok{4}\NormalTok{,}\DecValTok{5}\NormalTok{))}
\end{Highlighting}
\end{Shaded}

\begin{verbatim}
## [1] 0.3333333
\end{verbatim}

\begin{Shaded}
\begin{Highlighting}[]
\FunctionTok{Prob}\NormalTok{(S,X1 }\SpecialCharTok{\textgreater{}}\DecValTok{2}\NormalTok{)}
\end{Highlighting}
\end{Shaded}

\begin{verbatim}
## [1] 0.6666667
\end{verbatim}

Another way of doing this is by using events, that is describing events
as eventA , eventB, eventC and finding various combinations using these
events.

\begin{Shaded}
\begin{Highlighting}[]
\NormalTok{deck }\OtherTok{\textless{}{-}} \FunctionTok{cards}\NormalTok{(}\AttributeTok{makespace =} \ConstantTok{TRUE}\NormalTok{)}

\NormalTok{event1 }\OtherTok{\textless{}{-}} \FunctionTok{subset}\NormalTok{(deck, rank }\SpecialCharTok{\%in\%} \FunctionTok{c}\NormalTok{(}\StringTok{"Q"}\NormalTok{,}\StringTok{"K"}\NormalTok{))}
\CommentTok{\# whats the prob of drawing a queen or king}
\FunctionTok{Prob}\NormalTok{(event1)}
\end{Highlighting}
\end{Shaded}

\begin{verbatim}
## [1] 0.1538462
\end{verbatim}

\begin{Shaded}
\begin{Highlighting}[]
\CommentTok{\# prob of drawing a card between 3{-}5 and is also a heart?}
\FunctionTok{Prob}\NormalTok{(deck, rank }\SpecialCharTok{\%in\%} \FunctionTok{c}\NormalTok{(}\DecValTok{3}\NormalTok{,}\DecValTok{5}\NormalTok{,}\DecValTok{4}\NormalTok{) }\SpecialCharTok{\&}\NormalTok{ suit }\SpecialCharTok{==}\StringTok{"Heart"}\NormalTok{)}
\end{Highlighting}
\end{Shaded}

\begin{verbatim}
## [1] 0.05769231
\end{verbatim}

\begin{Shaded}
\begin{Highlighting}[]
\CommentTok{\# whats the intersection of these two event?}
\NormalTok{event2 }\OtherTok{\textless{}{-}} \FunctionTok{subset}\NormalTok{(deck,rank }\SpecialCharTok{\%in\%} \FunctionTok{c}\NormalTok{(}\DecValTok{3}\NormalTok{,}\DecValTok{5}\NormalTok{,}\DecValTok{4}\NormalTok{) }\SpecialCharTok{\&}\NormalTok{ suit }\SpecialCharTok{==}\StringTok{"Heart"}\NormalTok{)}
\FunctionTok{Prob}\NormalTok{(}\FunctionTok{union}\NormalTok{(event1,event2))}
\end{Highlighting}
\end{Shaded}

\begin{verbatim}
## [1] 0.2115385
\end{verbatim}

\hypertarget{conditional-probability}{%
\subsection{Conditional Probability}\label{conditional-probability}}

\begin{verbatim}
The conditional probability rule, P(B|A), computes the probability of the event B given that the event  A has
occurred.

P(B|A) = P(A ∩ B) / P(A)
\end{verbatim}

To demonstrate conditional probability, consider rolling two dice

\begin{Shaded}
\begin{Highlighting}[]
\CommentTok{\# follow the steps from above}
\CommentTok{\# 1. make a space}
\CommentTok{\# 2. Use the Prob() }
\NormalTok{S }\OtherTok{\textless{}{-}} \FunctionTok{rolldie}\NormalTok{(}\DecValTok{2}\NormalTok{,}\AttributeTok{makespace =} \ConstantTok{TRUE}\NormalTok{)}
\FunctionTok{head}\NormalTok{(S,}\AttributeTok{n=}\DecValTok{2}\NormalTok{)}
\end{Highlighting}
\end{Shaded}

\begin{verbatim}
##   X1 X2      probs
## 1  1  1 0.02777778
## 2  2  1 0.02777778
\end{verbatim}

\begin{Shaded}
\begin{Highlighting}[]
\CommentTok{\# Whats the prob of both die are equal}
\FunctionTok{Prob}\NormalTok{(S,X1 }\SpecialCharTok{==}\NormalTok{ X2)}
\end{Highlighting}
\end{Shaded}

\begin{verbatim}
## [1] 0.1666667
\end{verbatim}

\begin{Shaded}
\begin{Highlighting}[]
\CommentTok{\# Whats the prob of both equal given that they add up to 8}
\FunctionTok{Prob}\NormalTok{(S,X1}\SpecialCharTok{==}\NormalTok{X2, }\AttributeTok{given =}\NormalTok{ X1}\SpecialCharTok{+}\NormalTok{X2}\SpecialCharTok{==}\DecValTok{8}\NormalTok{)}
\end{Highlighting}
\end{Shaded}

\begin{verbatim}
## [1] 0.2
\end{verbatim}

\begin{Shaded}
\begin{Highlighting}[]
\CommentTok{\# whats the prob that they equal 8 given that they are equal}
\FunctionTok{Prob}\NormalTok{(S,X1}\SpecialCharTok{+}\NormalTok{X2}\SpecialCharTok{==}\DecValTok{8}\NormalTok{, }\AttributeTok{given =}\NormalTok{ X1}\SpecialCharTok{==}\NormalTok{X2)}
\end{Highlighting}
\end{Shaded}

\begin{verbatim}
## [1] 0.1666667
\end{verbatim}

\begin{Shaded}
\begin{Highlighting}[]
\CommentTok{\# You can describe these are Event A and Event B}
\NormalTok{A }\OtherTok{\textless{}{-}} \FunctionTok{subset}\NormalTok{(S, X1 }\SpecialCharTok{==}\NormalTok{ X2)}
\NormalTok{B }\OtherTok{\textless{}{-}} \FunctionTok{subset}\NormalTok{(S, X1 }\SpecialCharTok{+}\NormalTok{ X2 }\SpecialCharTok{==} \DecValTok{8}\NormalTok{)}

\FunctionTok{Prob}\NormalTok{(A, }\AttributeTok{given =}\NormalTok{ B)}
\end{Highlighting}
\end{Shaded}

\begin{verbatim}
## [1] 0.2
\end{verbatim}

\begin{Shaded}
\begin{Highlighting}[]
\FunctionTok{Prob}\NormalTok{(B, }\AttributeTok{given =}\NormalTok{ A)}
\end{Highlighting}
\end{Shaded}

\begin{verbatim}
## [1] 0.1666667
\end{verbatim}

Now lets try this with coin tosses!

\begin{Shaded}
\begin{Highlighting}[]
\NormalTok{S }\OtherTok{\textless{}{-}} \FunctionTok{tosscoin}\NormalTok{(}\DecValTok{2}\NormalTok{,}\AttributeTok{makespace =} \ConstantTok{TRUE}\NormalTok{)}
\NormalTok{S}
\end{Highlighting}
\end{Shaded}

\begin{verbatim}
##   toss1 toss2 probs
## 1     H     H  0.25
## 2     T     H  0.25
## 3     H     T  0.25
## 4     T     T  0.25
\end{verbatim}

\begin{Shaded}
\begin{Highlighting}[]
\CommentTok{\# Find any HEAD in two tosses of a coin}
\FunctionTok{Prob}\NormalTok{(S, }\FunctionTok{isin}\NormalTok{(S,}\FunctionTok{c}\NormalTok{(}\StringTok{"H"}\NormalTok{)))}
\end{Highlighting}
\end{Shaded}

\begin{verbatim}
## [1] 0.75
\end{verbatim}

\begin{Shaded}
\begin{Highlighting}[]
\CommentTok{\# Find any Head or Tails during two tosses}
\FunctionTok{Prob}\NormalTok{(S, }\FunctionTok{isin}\NormalTok{(S,}\FunctionTok{c}\NormalTok{(}\StringTok{"H"}\NormalTok{,}\StringTok{"T"}\NormalTok{)))}
\end{Highlighting}
\end{Shaded}

\begin{verbatim}
## [1] 0.5
\end{verbatim}

\begin{Shaded}
\begin{Highlighting}[]
\NormalTok{A }\OtherTok{\textless{}{-}} \FunctionTok{subset}\NormalTok{(S,}\FunctionTok{isin}\NormalTok{(S,}\FunctionTok{c}\NormalTok{(}\StringTok{"H"}\NormalTok{)))}
\NormalTok{B }\OtherTok{\textless{}{-}} \FunctionTok{subset}\NormalTok{(S,}\FunctionTok{isin}\NormalTok{(S,}\FunctionTok{c}\NormalTok{(}\StringTok{"H"}\NormalTok{,}\StringTok{"T"}\NormalTok{)))}
\CommentTok{\# Prob of finding a H given H and a T was found}
\FunctionTok{Prob}\NormalTok{(A, }\AttributeTok{given =}\NormalTok{ B)}
\end{Highlighting}
\end{Shaded}

\begin{verbatim}
## [1] 1
\end{verbatim}

\begin{Shaded}
\begin{Highlighting}[]
\FunctionTok{Prob}\NormalTok{(B, }\AttributeTok{given =}\NormalTok{ A)}
\end{Highlighting}
\end{Shaded}

\begin{verbatim}
## [1] 0.6666667
\end{verbatim}

\hypertarget{using-urnsamples}{%
\subsection{Using URNSAMPLES}\label{using-urnsamples}}

Define an URN that takes in one six sided die

\begin{Shaded}
\begin{Highlighting}[]
\FunctionTok{urnsamples}\NormalTok{(}\DecValTok{1}\SpecialCharTok{:}\DecValTok{6}\NormalTok{,}\AttributeTok{size =} \DecValTok{2}\NormalTok{)}
\end{Highlighting}
\end{Shaded}

\begin{verbatim}
##    X1 X2
## 1   1  2
## 2   1  3
## 3   1  4
## 4   1  5
## 5   1  6
## 6   2  3
## 7   2  4
## 8   2  5
## 9   2  6
## 10  3  4
## 11  3  5
## 12  3  6
## 13  4  5
## 14  4  6
## 15  5  6
\end{verbatim}

\end{document}
